\section{Introduction}

\cite{Tamrawi2012}

Project build management is important.
Build configuration files, like Makefile becomes complex as the project grows.
Build configuration files are hard to understand by human beings.
Hard to debug, not interactive debugger.
Hard to maintain, oftentimes developers keep adding new rules, but old rules
remain in makefiles. They don't have confidence to delete old rules.

Visualization can help understand build management. 
There are some projects that visualize build configuration files.
They visualize dependency graph.
But most of them rely on static analysis of makefile.
This doesn't work quite well, build process is dynamic, and highly depends on
runtime environment.
Complex implicit rules, weird syntax, for loops/condition branches ...

visualize what would really happen at runtime. 
let build program parse configuration files, but don't really build the project.
we only want the real dependency graph and other related runtime infomation.
Can slightly modify build program, or better, build program verbose output.
Start with makefile, but can be generalize to other build system.
we do visualization in different styles for different application.

What would be done:
1. Trace data from make -p -n. 
2. Visualize in 3 styles using d3: network, chord, indented tree
   - http://bl.ocks.org/1153292 (network) -- incremental development, small projects
   - http://www.tips-for-excel.com/MCFC/Passes.html (chord) -- dependency
   - http://bl.ocks.org/1093025 (indented tree) -- component, hierarchy
====> 1 + 2: a usable tool, vizmake -> automated visualization in browser
3. Case study on real systems (e.g., linux, mysql, condor ...): debugging,
   refactoring, component analysis
4. More: find real bugs, real refactoring, interaction (filter, query,
   annotation, related ...)
